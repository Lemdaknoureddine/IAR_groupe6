\documentclass[paper=a4, fontsize=11pt]{scrartcl}
\usepackage[T1]{fontenc}
\usepackage{fourier}
\usepackage{ graphicx }
%\usepackage[english]{babel}															
\usepackage[english]{babel}
\usepackage{indentfirst}		% for indent
\usepackage[utf8]{inputenc}


\usepackage[protrusion=true,expansion=true]{microtype}	
\usepackage{amsmath,amsfonts,amsthm} % Math packages
\usepackage[pdftex]{graphicx}	
\usepackage{url, array}
\usepackage[num,abnt-repeated-author-omit=yes]{abntex2cite}


%%% Custom sectioning
\usepackage{sectsty}
\allsectionsfont{\centering \normalfont\scshape}


%%% Custom headers/footers (fancyhdr package)
\usepackage{fancyhdr}
\pagestyle{fancyplain}
\fancyhead{}											% No page header
\fancyfoot[L]{}											% Empty 
\fancyfoot[C]{}											% Empty
\fancyfoot[R]{\thepage}									% Pagenumbering
\renewcommand{\headrulewidth}{0pt}			% Remove header underlines
\renewcommand{\footrulewidth}{0pt}				% Remove footer underlines
\setlength{\headheight}{13.6pt}


%%% Equation and float numbering
\numberwithin{equation}{section}		% Equationnumbering: section.eq#
\numberwithin{figure}{section}			% Figurenumbering: section.fig#
\numberwithin{table}{section}				% Tablenumbering: section.tab#


%%% Maketitle metadata
\newcommand{\horrule}[1]{\rule{\linewidth}{#1}} 	% Horizontal rule

%\date{\today}

%%% Begin document
\begin{document}
		
{\flushleft\horrule{2pt}
\begin{center}
{\includegraphics[height=0.09\textwidth]{logo_english.png}} 
\begin{tabular}{ m{1.8cm} m{10cm} m{1.8cm}}
\begin{center}
\end{center}
&
\begin{center} 
{\small
{Adrar University} \\
{Faculty of Science and Technology} \\
{Department of Mathematics and Computer Science}} \\

\end{center}
&

\begin{center}
\end{center}
\end{tabular}
\end{center}
\flushleft \horrule{2pt}\\[1cm]
}


\begin{center}

{
\huge  
Initiation to Research (Course) \\
\vspace{0.2cm}
2\textsuperscript{nd} Year Master (S3) \\
\vspace{0.2cm}
2020/2021}\\

\vspace{1cm}

{
\Huge   
\textbf{la communication entre réseau de capteur sans fil  }}\\
\vspace{1cm}

%Domain: Mathematics and Computer Science \\
%Major: Informatics - Intelligent Systems \\
{
\Large
\textbf{lemdak noureddine \footnote{lemdaknoureddine01@gmail.com} \\Idrissi mohammed \footnote{Idrissi.mohammed1999@gmail.com} \\ missirene abdelkarim \footnote{missirenekarim@gmail.com}}}\\
\vspace{3cm}

{
\large
\textbf{Instructor: Dr. Abdelghani DAHOU \footnote{Email: dahou.abdghani@univ-adrar.edu.dz}}}\\
\today
\end{center}
\pagebreak
\tableofcontents
\pagebreak
\listoffigures
\textbf{Figure 2.1 :} Capteur sans fils .\\
\textbf{Figure 2.2 :} Architecture d'un capteur sans fil .\\
\textbf{Figure 2.3 :} Exemple d’un réseau de capteur.\\
\pagebreak





...



\section{Réisme }

Dans les communications sans fil, les réseaux de capteurs sans fil représentent une
solution difficile et distante, et cela représente un défi dans la conception de tels
réseaux, en particulier en ce qui concerne la composante énergétique, car la batterie
doit durer toute la vie du capteur, du RCSF et la qualité et la fiabilité des données
capturées à partir du champ du capteur et assurer qu'elles arrivent au terminal de données. \par
Nous mettrons en œuvre le protocole TEEN pour trouver des solutions pour
surmonter ces limitations.
\end{itemize}

\section{Introduction}
En communication sans fil, le développement de la technologie dans le domaine de
l'information peut conduire à la production de capteurs intelligents appelés (nœuds),
où ils sont placés dans une vaste zone appelée le champ des capteurs, qui sont reliés
entre eux pour former un capteur sans fil réseau, qui nous aide à surveiller les
phénomènes naturels dans des endroits éloignés ou même à prédire avant les
catastrophes, tels que les phénomènes géologiques des tremblements de terre,
l'activité volcanique, ou la surveillance des variables astronomiques, ainsi que la
médecine et le domaine militaire, en général, des solutions aux difficiles, zones
dangereuses et isolées. \par
Le nœud de capteur se compose de quatre unités: l'unité réceptrice (radio), l'unité
émettrice, l'unité de traitement et l'unité d'alimentation, chacune ayant son rôle de
base et son importance dans la construction d'un réseau solide, les données sont
envoyées au reste de les nœuds ou à la station de base pour organiser et traiter les
informations, cela peut être une station Une autre base fixe ou mobile, capable de
connecter le réseau de capteurs avec une infrastructure avancée ou Internet, où
l'utilisateur peut consulter.\par
Le réseau de capteurs sans fil vise souvent à établir des protocoles permettant
d'étendre les nœuds dans des environnements difficiles et difficiles à atteindre sans
aucune intervention humaine. C'est le problème auquel ces nœuds sont confrontés.
En utilisant la science et la technologie modernes, ils ont développé un ensemble de
protocoles spécialement conçus pour les réseaux de capteurs sans fil, qui aident à
prolonger au maximum la durée de vie des nœuds tout en garantissant la qualité des
données.\par
Le but de ce projet est de mettre en œuvre le protocole de routage hiérarchique basé
sur des blocs TEEN pour voir l'efficacité de l'augmentation de la durée de vie du
réseau de capteurs sans fil.

\section{Travaux connexes}
\section{Capteur }
\subsection{Dfinition}
Un capteur sans fil est un petit dispositif électronique capable de mesurer une valeur
physique environnementale et de la communiquer à un centre de contrôle via une station de
base [1]. \par

    \begin{center}
    \includegraphics[width=6cm]{cqpteur.jpg} \par
    \caption{  \textbf{Figure 2.1 :} Capteur sans fils } \par
   \end{center}
Les capteurs sont dotés d’une batterie, capables de communiquer entre eux et de surveiller
une grande variété de phénomènes ambiants, notamment : la température, l’humidité, la
pression, le taux de bruits, la présence ou pas des certains types d’objets, et d’autres
caractéristiques, tel que la vitesse, la direction et le volume d’un objet donné et de les
transformer en données numériques, afin de les communiquer par ondes radio à travers le
réseau vers la station de base (SB).\par
\section{Architecteur de nœud capteur }
\subsection{ Architecture matérielle}
\begin{center}
    \includegraphics[width=6cm]{ar.png} \par
    \caption{  \textbf{Figure 2.2 :} Architecture d'un capteur sans fil } \par
    \end{center}
Cette architecture est basée sur quatre unités:\par
\textbf{Unité de traitement:}Il s'agit de l'unité principale du capteur. Il contient un
processeur couplé à une mémoire vive. Son rôle est de contrôler le bon
fonctionnement des autres unités. Dans certains capteurs, ils peuvent avoir un
système d'exploitation embarqué pour faire fonctionner le capteur, et pour
enregistrer les informations envoyées par l'unité d'acquisition de données, celle-ci
est couplée au volume.\par
\textbf{ Unité d'acquisition (unité de capteur):} elle capte ou détecte l'événement du champ
de détection et le convertit des mesures physiques ou analogiques en données
numériques. Il se compose du capteur lui-même et d'un ADC (convertisseurs
analogique-numérique) qui permet les données à convertir.\par
\textbf{ Unité de communication (unité d'émission et de réception):} Sa fonction est de
transmettre et de recevoir des informations. Il est équipé d'une paire émetteur /
récepteur pour communiquer au sein du réseau. Cependant, il existe d'autres
possibilités de transmission (optique, infrarouge, etc.).\par
 \textbf{Bloc d'alimentation}: il s'agit d'un composant essentiel de l'architecture du
capteur, et c'est celui qui alimente toutes les autres unités. Il correspond souvent à
une batterie ou une cellule qui alimente le capteur, et ses ressources limitées en font
un problème propre à ce type de réseau car celui-ci est généralement déployé dans
des zones inaccessibles.
La dernière réalisation d'un bloc d'alimentation basé sur un panneau solaire tente de
fournir une solution pour prolonger sa durée de vie [2].
De plus, d'autres unités peuvent être équipées d'un capteur. Comme un système de
positionnement global (GPS), une unité de navigation pour assurer le mouvement du
capteur, ou une unité de capture spécifique telle qu'une caméra pour acquérir la
vidéo.\par
\subsection{Architecture Logicielle}
La limitation de puissance du capteur nécessite l'utilisation de systèmes OS léger
comme TinyOS ou Contiki, cependant, TinyOS est toujours le plus utilisé et le plus
populaire dans l'arène RCSF. Il est gratuit et une grande communauté de
scientifiques l'utilise dans des simulations pour développer et tester des algorithmes
et des protocoles de réseau.\par
\textbf{ Système d'exploitation TinyOS :}TinyOS est un système d'exploitation open source développé et maintenu par
l'Université de Berkeley. Ce système d'exploitation est conçu pour les réseaux de
capteurs sans fil car le capteur ne dispose pas de suffisamment de mémoire pour
prendre en charge un système d'exploitation comme Linux ou Windows qui prend
beaucoup de mémoire.
TinyOS a été créé pour répondre aux caractéristiques et aux exigences des réseaux
de capteurs, tels que [3]:
-Réduction de la taille de la mémoire\par
-Faible consommation d'énergie.\par
- Opérations auxiliaires intenses et puissantes.\par
- Il a été amélioré en termes de mémoire et d'utilisation d'énergie.\par
TinyOS est un système d'exploitation basé sur
Événement; Cela signifie qu'il devient actif lorsqu'un événement se produit pour
mieux fournir les ressources énergétiques des capteurs.\par
\textbf{ Système d'exploitation Contiki :}
Contiki OS est un système d'exploitation développé par Dunkel et d'autres. Contiki
OS, qui est basé sur le langage de programmation open-source C, a été développé
pour les réseaux de capteurs sans fil légers, flexibles et de faible puissance.
Les environnements WSN sont souvent limités en puissance, comme mentionné.
C'est l'une des principales limites du RCSF. De même, les conceptions de nœuds
petits et simples sont les autres limitations. Pour cette raison, RCSF est un must have
Les fonctions matérielles et logicielles critiques pour gérer ces nœuds. Contiki OS
est une solution pratique pour contrer les limitations susmentionnées grâce à sa
flexibilité et à sa prise en charge des réseaux minces et de faible puissance.\par
\section{ Réseau de capteur sans fil}
\subsection{Définition}
Le réseau de capteurs sans fil (RCSF), le Wireless sensor Network (WSN), se compose d'un
grand nombre de nœuds de capteurs intelligents, «Smart sensor», numérotés de dizaines à
plusieurs milliers d'éléments. Dans les sous-réseaux distribués pour collecter des
informations du monde physique dans dont il est publié. Ensuite, les données collectées sont
envoyées à une station de traitement de données appelée " Station de Base " (BS: Base
Station).\par
Ce transfert des données collectées peut être effectué, périodiquement ou sur une base
événementielle, vers la station de base directement ou via un ou plusieurs nœuds privés
fixes ou mobiles appelés «bassins».\par
Dans ces réseaux, chaque nœud est capable de surveiller son environnement et de répondre
si nécessaire en envoyant les informations collectées à un ou plusieurs points de collecte, et
les données capturées sont acheminées via un routage multi-hop vers un nœud appelé "point
de collecte" du nœud de bassin [4].\par
\begin{center}
    \includegraphics[width=8cm]{rs.png} \par
    \caption{  \textbf{Figure 2.3 :} Exemple d’un réseau de capteur. } \par
    \end{center}
\subsection{Les composants de réseau de capteurs sans fil}
Les réseaux de capteurs sans fil (WSN) sont composés de nœuds, qui sont des unités
de capteurs sans fil, un routeur et une station de base.\par
\textbf{ Nœuds de capteur :}Un capteur est un appareil qui détecte des informations et les transforme en mot.
Le dispositif comprend un processeur, une mémoire, une batterie, un émetteur et un
récepteur pour former un réseau ad hoc.\par
 \textbf{Station de base :}
Une station (composée d'un processeur, d'une carte radio, d'une antenne et d'une
carte d'interface USB) connecte le réseau de capteurs à un autre réseau. ll est
préprogrammé avec un logiciel de réseau basse consommation pour communiquer
avec des nœuds de capteurs sans fil.\par
 \textbf{Le routeur :}
Un routeur est un appareil contrôlé par un microprocesseur connecté entre deux
types de réseaux différents. Il transfère également des paquets de données entre les
réseaux informatiques. Le routeur est utilisé pour se connecter à divers réseaux;
Extrait la destination du colis et sélectionne la meilleure destination.
Le routeur détermine la destination du paquet et les informations qu'il contient.
En utilisant les informations de la table de routage, le paquet est acheminé vers le
réseau suivant.\par 
\subsection{Caractéristiques des réseaux de capteurs sans fil }
Un réseau de capteurs sans fil présente de nombreuses caractéristiques importantes,
notamment:\par
\textbf{Durée de vie limitée :}
 Les capteurs utilisent leurs énergies pour calculer et transmettre des données.
Chaque nœud joue le rôle d'un émetteur et d'un routeur, de sorte qu'une coupure de
courant dans le nœud de détection peut affecter des changements majeurs dans la
topologie du réseau et nécessiter une réorganisation coûteuse de ce dernier [5].\par
\textbf{Ressources limitées :}
Le facteur de forme (très petit) limite la quantité de ressources qui peuvent être
mises dans ces nœuds, donc la capacité de traitement et la mémoire sont très limitées.\par.
\textbf{Bande passante limitée (supports de transmission) :}
Pour la communication entre les nœuds, différents supports sans fil sont utilisés
(radio, infrarouge, optique), en fonction de l'environnement dans lequel ils se
trouvent et de la tâche qui leur est demandée, ils doivent donc être compatibles.
Cependant, la majorité des capteurs communiquent via l'utilisation d'un circuit RF.
En raison de la puissance limitée, les nœuds de capteurs ne sont pas capables de
supporter des débits très élevés [5].\par
\textbf{Évolutivité :}
Le nombre de nœuds de capteurs dans le réseau de capteurs peut être de l'ordre de
centaines ou de milliers (selon l'application) [5].
\textbf{Structure dynamique :}
La topologie des réseaux de capteurs change fréquemment et rapidement car les
nœuds peuvent être déployés dans des environnements difficiles (par ex.
Exemple Battlefield), ainsi qu'une défaillance très probable du nœud de capteur.
Cependant, les nœuds de capteur et les points de terminaison (nœuds de destination)
où les informations capturées sont envoyées peuvent être mobiles, de sorte que le
transport de messages vers ou vers un nœud mobile est un autre défi. Par
conséquent, la capture peut être à la fois statique et dynamique en fonction de
l'application [5].\par
\textbf{ Collecte de données :}
Les techniques de collecte de données concernent le traitement des données par le
réseau et la réduction du nombre de messages .
Réduisant ainsi la consommation d'énergie. Dans les RCSF, les données produites
par les nœuds de capteurs sont fortement corrélées, ce qui signifie qu'il existe une
redondance dans les données. Les utilisateurs s'intéressent au phénomène .
Capturés par les données générées par chaque nœud, ces réseaux offrent ainsi la
possibilité d'agréger des données pour réduire la bande passante [5].\par
 \textbf{Réseau autorégulé :}
Il est pratiquement impossible de réaliser une configuration manuelle du réseau en
raison du grand nombre de nœuds et de leurs déploiements dans
Environnements hostiles. De plus, les nœuds peuvent quitter le réseau avec une
panne (coupure de courant, panne physique, etc.), et d'autres peuvent les fusionner,
il est donc nécessaire que le réseau s'organise de lui-même [5].\par
\subsection{Application RCSF}
Nous catégorisons les implémentations RCSF en quatre catégories: basées sur le
temps (basées sur le temps), sur les événements (sur les événements), sur les
requêtes (sur les requêtes) et hybrides [6].\par
\textbf{ Applications temporelles :}
Cette catégorie représente des applications où l'acquisition et la transmission des
données capturées sont liées au temps, la quantité de données échangées dans le
réseau dépend de la périodicité des mesures qui seront faites sur l'environnement
local, par exemple dans les domaines: agricole, scientifique expériences, etc.\par
\textbf{ Applications orientées événements :}
Dans ce cas, les capteurs envoient leurs données uniquement dans le cas où un
certain événement se produit. Exemple de surveillance des feux de forêt où le
capteur envoie des alarmes à la station de base dès que la température dépasse un
certain seuil.\par
\textbf{Demandes adressées :}
Dans ce cas, le capteur n'envoie les informations qu'après une demande explicite de
la station de base. Cette catégorie d'applications est destinée aux applications
conviviales. Ces derniers peuvent demander des informations à des zones
spécifiques du réseau ou interroger des capteurs pour obtenir des métriques d'intérêt.
Dans ce cas, la connaissance de la topologie du réseau et de l'emplacement des
capteurs est requise.\par
\textbf{ Applications hybrides :}
Cette application est basée sur un mélange des types mentionnés précédemment. Par
exemple, dans un réseau conçu pour suivre des objets, le réseau peut combiner un 
réseau de surveillance (basé sur le temps) avec un réseau de collecte de données
basé sur les événements. Par exemple, pendant des périodes prolongées d'inactivité
du capteur et en l'absence d'objet, le réseau peut assurer une fonction de
surveillance.

\p\end{itemize}

\section{Méthodes de recherche}
\subsection{Routage}
Le routage est une méthode d’acheminement des informations vers la bonne
destination à travers un réseau de connexion donné, il consiste à assurer unestratégie
qui garantit, n’importe quel moment, un établissement de routes correcteset efficaces
entre n’importe Quelle paire de noeuds appartenant au réseau.
Ce qui assure l’échange des messages d’une manière continue. Vu les limitations
des réseaux ad hoc, la construction des routes doit être faite avec un minimum de
contrôle et de consommation de bande passante. [7]\par
\subsection{Routage hiérarchique }
Les méthodes de routage hiérarchique ont des avantages spéciaux liés au ‘passage à
l’échelle et à l’efficacité dans la communication. Par exemple, elles sont utilisées
pour exécuter un routage avec économie d’énergie dans les RCSF.
Dans une architecture hiérarchique, des noeuds à grande énergie peuvent être
employés pour traiter et envoyer l’information, alors que des noeuds à énergie
réduite peuvent assurer la capture à proximité de la cible. La création des clusters et
l’assignation des tâches spéciales aux têtes de clusters peuvent considérablement
renforcer le passage à l’échelle, l’augmentation de la durée de vie et l’efficacité
énergétique du système global.\par
Le routage hiérarchique est une manière efficace de réduire la consommation
énergétique dans un cluster en exécutant l’agrégation et la fusion de données afin de
diminuer le nombre de messages transmis à la station de base. [8]\par
Les nœuds à faible énergie peuvent être employés pour exécuter la tâche de la
capture à proximité de la cible .
\par
\subsection{ Caractéristique d'un protocole de routage hiérarchique }
Un protocole de routage hiérarchique doit s’écrier plusieurs tâches mais tout d'abord
quelques dé nations s’imposent.\par
Le protocole de routage hiérarchique est associé à de nombreuses tâches, mais avant tout,
certains défis émergent.\par
\textbf{ Clustering:} Le Clustering est une technique permettant de diviser le réseau en groupes
(clusters), sachant que pour chaque groupe un leader est affecté à chaque groupe (Cluster Head)
pour communiquer avec les éléments de son groupe, le processus d'assemblage contribue
grandement à économiser l'énergie , réduisant la complexité des protocoles de routage et la
flexibilité du facteur de mesure En plus de la collecte de données, qui permet d'éliminer la
redondance des données et d'envoyer uniquement les informations utiles\par
\textbf{ Cluster:} un cluster est un groupe de nœuds qui forment l'unité organisationnelle du réseau
de capteurs.\par
\textbf{Cluster Head : }il représente le chef de groupe et a diverses tâches telles que l'organisation
de la communication entre les groupes et au sein du groupe, la collecte des données, et il est élu
par d'autres nœuds ou il peut être désigné à l'avance par le concepteur du réseau.
\section{ TEEN (Threshold-sensitive Energy Efficient Sensor Network protocole)}
Manjeshwar et Agrawal [9] ont proposé une technique de clustering appelée TEEN
pour les applications critiques où le changement de certains paramètres peut être
brusque. L’architecture du réseau est basée sur un groupement hiérarchique à
plusieurs niveaux où les noeuds les plus proches forment des clusters.\par
Puis ce processus de clustering passe au deuxième niveau jusqu’à ce que la station
de base soit atteinte. Après la formation des clusters, chaque cluster head transmet à
ses membres deux seuils : un seuil Hard HT (hard threshold), qui est la valeur seuil
du paramètre contrôlé (surveillé) et un seuil Soft ST (soft threshold) représentant
une petite variation de la valeur du paramètre contrôlé.\par
L’occurrence de cette petite variation ST permet au noeud qui la détecte de la
signaler à la station de base en transmettant un message d’alerte. Par conséquent, le
seuil Soft réduira le nombre de transmissions puisqu’il ne permet pas la transmission
s’il y a peu ou pas de variation de la valeur du paramètre contrôlé.\par
Au début, les nœuds écoutent le médium continuellement et lorsque la valeur captée
du paramètre contrôlé dépasse le seuil Hard, le noeud transmet l’information. La
valeur captée est stockée dans une variable interne appelée SV.\par
Puis, les noeuds ne transmettront des données que si la valeur courante du paramètre
contrôlé est supérieure au seuil hard HT ou diffère du SV d’une quantité égale ou
plus grande que la valeur du seuil Soft ST.\par
Puisque la transmission d’un message consomme plus d’énergie que la détection des
données, alors la consommation d’énergie dans TEEN est moins importante que
dans les protocoles proactifs ou ceux qui transmettent des données périodiquement
tels que LEACH.
\end{itemize}

\pagebreak
\bibliographystyle{abnt-num}
\bibliography{ref}
[1] K. Beydoun.“ Conception d’un protocole de routage hiérarchique pour les
réseaux de capteurs, " Thèse de Doctorat en informatique, Université de FrancheComté des sciences et techniques,2009.\\
\bibliography{ref}
[2] David Martins, ”Sécurité dans les réseaux de capteurs sans fil Sténographie et réseaux
de confiance“, L'U.F.R. des Sciences et Techniques de l’université de Franche-Comté, 2010.\\  \bibliography{ref}
[3] Bouzidi Zeyneb et Benameur Amina, ”Mise en place d’un réseau de capteurs sans fil
pour l’irrigation intelligente“, Mémoire de Master, Université de Tlemcen, 2012.\\ \bibliography{ref}
[4] K.B. Kredo, B.P. Mohapatra. "Medium access control in wireless sensor networks",
Computer network 51(4), pp 961-994 , 2007.\\ \bibliography{ref}
[5] MERRANI Nassima, KHIMOUM Nadia. ’ Simulation et évaluation de protocoles de
gestion de clés dans les réseaux de capteurs’. Mémoire d’ingénieur d’état en informatique.
Bejaia 2009.\\ \bibliography{ref}
[6] M. Ilyas and I. Mahgoub. "Handbook of sensor networks Compact wireless and wired
Sensing Systems", ISBN 08493196864. CRC PRESS LLS, USA, 2005.\\ \bibliography{ref}
[7] Mr RAHMOUNE Amer, “Simulation d’un protocole de surveillance des interfaces d’un
routeur“, mémoire présenté pour obtenir le titre d'ingénieur d'état, Université A/MIRA de
Bejaïa, 2014/2015.\\ \bibliography{ref}
[8] Meldjem Sara , “,Les réseaux de capteurs , mémoire présenté pour obtenir le Licence 3 :
GTR, Université des sciences et de la Technologie Houari Boumediene,2013/2014.\\ \bibliography{ref}
[9] D.P. Agrawal A. Manjeshwar. “TEEN : a routing protocol for enhanced efficiency in
wireless sensor networks“. Proceedings 15th International Parallel and Distributed
Processing Symposium, 2001.

\end{document}